% options for packages loaded elsewhere
\PassOptionsToPackage{unicode=true}{hyperref}
\PassOptionsToPackage{hyphens}{url}

% specify apa6 document mode with command to have a default
\newcommand{\pandocDocMode}{man}

% apa6 mode and class options
\documentclass[\pandocDocMode,longtable, floatsintext, noextraspace]{apa6}
% \documentclass[
%   %     \pandocDocMode,longtable
%   %   
% ]{apa6}

% for mode selection options
\usepackage{ifthen}

\newcommand{\forceLongTablePkg}{}

% % setup mode ifs
% \newif\ifmanmode
% \newif\ifdocmode
% \newif\ifjoumode
% \ifthenelse{\equal{\string \pandocDocMode}{\string man}}{
%     \manmodetrue
% }{
%     \renewcommand{\forceLongTablePkg}{longtable}
%     \ifthenelse{\equal{\string \pandocDocMode}{\string doc}}{
%         \docmodetrue
%     }{
%         \ifthenelse{\equal{\string \pandocDocMode}{\string jou}}{
%             \joumodetrue
%         }{
% % None
% }}}


\usepackage{\forceLongTablePkg}

% % % man mode uses the endfloat package,
% % this will try to set tables and figures at the end for longtables too
% \ifmanmode
% \DeclareDelayedFloatFlavor{longtable}{table}
% \fi
% 
% other packages
\usepackage{lmodern}
\usepackage{amsmath,amssymb}
\usepackage{ifxetex,ifluatex}
\usepackage{fixltx2e} % provides \textsubscript

% handle different types of tex engines
% if pdftex
\ifnum 0\ifxetex 1\fi\ifluatex 1\fi=0
  \usepackage[T1]{fontenc}
  \usepackage[utf8]{inputenc}
  \usepackage{textcomp} % provides euro and other symbols
% if luatex or xelatex
\else
  \usepackage{unicode-math}
  \defaultfontfeatures{Ligatures=TeX,Scale=MatchLowercase}
\fi

% other language options
\usepackage[american]{babel}
\usepackage{csquotes}
\usepackage{microtype}

% disable microtype protrusion for tt fonts
\UseMicrotypeSet[protrusion]{basicmath}

\usepackage{graphicx,grffile}
% Scale images if necessary, so that they will not overflow the page
% margins by default, and it is still possible to overwrite the defaults
% using explicit options in \includegraphics[width, height, ...]{}
\makeatletter
\def\maxwidth{\ifdim\Gin@nat@width>\linewidth\linewidth\else\Gin@nat@width\fi}
\def\maxheight{\ifdim\Gin@nat@height>\textheight\textheight\else\Gin@nat@height\fi}
\makeatother
\setkeys{Gin}{width=\maxwidth,height=\maxheight,keepaspectratio}

\usepackage{booktabs}
\usepackage{caption}
\usepackage{subcaption}

% Pandoc stuff
\let\tightlist\relax % empty pandoc tight list command

% Hyperlinks and other metadata in pdf
\usepackage{hyperref}
\hypersetup{
            pdftitle={Exception Handling and File Processing},
            pdfauthor={Thomas Culpepper},
            colorlinks=false,
            linkcolor=Black,
            citecolor=Black,
            urlcolor=Black,
            pdfborder={0 0 0},
            breaklinks=true}
\urlstyle{same} % don't use monospace font for urls



\usepackage{color}
\usepackage{fancyvrb}
\newcommand{\VerbBar}{|}
\newcommand{\VERB}{\Verb[commandchars=\\\{\}]}
\DefineVerbatimEnvironment{Highlighting}{Verbatim}{commandchars=\\\{\}}
% Add ',fontsize=\small' for more characters per line
\newenvironment{Shaded}{}{}
\newcommand{\AlertTok}[1]{\textcolor[rgb]{1.00,0.00,0.00}{\textbf{#1}}}
\newcommand{\AnnotationTok}[1]{\textcolor[rgb]{0.38,0.63,0.69}{\textbf{\textit{#1}}}}
\newcommand{\AttributeTok}[1]{\textcolor[rgb]{0.49,0.56,0.16}{#1}}
\newcommand{\BaseNTok}[1]{\textcolor[rgb]{0.25,0.63,0.44}{#1}}
\newcommand{\BuiltInTok}[1]{#1}
\newcommand{\CharTok}[1]{\textcolor[rgb]{0.25,0.44,0.63}{#1}}
\newcommand{\CommentTok}[1]{\textcolor[rgb]{0.38,0.63,0.69}{\textit{#1}}}
\newcommand{\CommentVarTok}[1]{\textcolor[rgb]{0.38,0.63,0.69}{\textbf{\textit{#1}}}}
\newcommand{\ConstantTok}[1]{\textcolor[rgb]{0.53,0.00,0.00}{#1}}
\newcommand{\ControlFlowTok}[1]{\textcolor[rgb]{0.00,0.44,0.13}{\textbf{#1}}}
\newcommand{\DataTypeTok}[1]{\textcolor[rgb]{0.56,0.13,0.00}{#1}}
\newcommand{\DecValTok}[1]{\textcolor[rgb]{0.25,0.63,0.44}{#1}}
\newcommand{\DocumentationTok}[1]{\textcolor[rgb]{0.73,0.13,0.13}{\textit{#1}}}
\newcommand{\ErrorTok}[1]{\textcolor[rgb]{1.00,0.00,0.00}{\textbf{#1}}}
\newcommand{\ExtensionTok}[1]{#1}
\newcommand{\FloatTok}[1]{\textcolor[rgb]{0.25,0.63,0.44}{#1}}
\newcommand{\FunctionTok}[1]{\textcolor[rgb]{0.02,0.16,0.49}{#1}}
\newcommand{\ImportTok}[1]{#1}
\newcommand{\InformationTok}[1]{\textcolor[rgb]{0.38,0.63,0.69}{\textbf{\textit{#1}}}}
\newcommand{\KeywordTok}[1]{\textcolor[rgb]{0.00,0.44,0.13}{\textbf{#1}}}
\newcommand{\NormalTok}[1]{#1}
\newcommand{\OperatorTok}[1]{\textcolor[rgb]{0.40,0.40,0.40}{#1}}
\newcommand{\OtherTok}[1]{\textcolor[rgb]{0.00,0.44,0.13}{#1}}
\newcommand{\PreprocessorTok}[1]{\textcolor[rgb]{0.74,0.48,0.00}{#1}}
\newcommand{\RegionMarkerTok}[1]{#1}
\newcommand{\SpecialCharTok}[1]{\textcolor[rgb]{0.25,0.44,0.63}{#1}}
\newcommand{\SpecialStringTok}[1]{\textcolor[rgb]{0.73,0.40,0.53}{#1}}
\newcommand{\StringTok}[1]{\textcolor[rgb]{0.25,0.44,0.63}{#1}}
\newcommand{\VariableTok}[1]{\textcolor[rgb]{0.10,0.09,0.49}{#1}}
\newcommand{\VerbatimStringTok}[1]{\textcolor[rgb]{0.25,0.44,0.63}{#1}}
\newcommand{\WarningTok}[1]{\textcolor[rgb]{0.38,0.63,0.69}{\textbf{\textit{#1}}}}



% Title page stuff
\title{Exception Handling and File Processing}
\shorttitle{Exception Handling and File Processing}
\author{Thomas Culpepper}
\affiliation{CSC310 Module 1 Case\\Trident University International}


% 
% 
% 

% \newcommand{\journalOptions}{
% \ifthenelse{\equal{\string \pandocDocMode}{\string jou}}{

% % Journal specific commands may go here:
% % % % % % }{}
% }

% \journalOptions{}


% table caption width
\makeatletter
\newcommand\LastLTentrywidth{1em}
\newlength\longtablewidth
\setlength{\longtablewidth}{1in}
\newcommand\getlongtablewidth{%
\begingroup
\ifcsname LT@\roman{LT@tables}\endcsname
\global\longtablewidth=0pt
\renewcommand\LT@entry[2]{\global\advance\longtablewidth by ##2\relax\gdef\LastLTentrywidth{##2}}%
\@nameuse{LT@\roman{LT@tables}}%
\fi
\endgroup}


%% pandoc-tablenos: environment to disable table caption prefixes
\makeatletter
\newcounter{tableno}
\newenvironment{tablenos:no-prefix-table-caption}{
  \caption@ifcompatibility{}{
    \let\oldthetable\thetable
    \let\oldtheHtable\theHtable
    \renewcommand{\thetable}{tableno:\thetableno}
    \renewcommand{\theHtable}{tableno:\thetableno}
    \stepcounter{tableno}
    \captionsetup{labelformat=empty}
  }
}{
  \caption@ifcompatibility{}{
    \captionsetup{labelformat=default}
    \let\thetable\oldthetable
    \let\theHtable\oldtheHtable
    \addtocounter{table}{-1}
  }
}
\makeatother


% PREAMBLE END
% --------------
% DOCUMENT START

\begin{document}
\maketitle

\hypertarget{mycode}{%
\label{mycode}}%
\begin{Shaded}
\begin{Highlighting}[numbers=left,,firstnumber=100,]
 \CommentTok{/**}
\CommentTok{*}\NormalTok{ The CalcTaxes program implements an application that requests tax information }
\CommentTok{*}\NormalTok{ from the user and then calculates taxes due}\CommentTok{.}
\CommentTok{*}
\CommentTok{*}\NormalTok{ Assignment}\CommentTok{:}\NormalTok{ CSC310 Mod }\CommentTok{1}\NormalTok{ Case}
\CommentTok{*}
\CommentTok{*} \CommentTok{@}\NormalTok{author  Thomas Culpepper}
\CommentTok{*} \CommentTok{@}\NormalTok{version }\CommentTok{1.0}
\CommentTok{*} \CommentTok{@}\NormalTok{since   }\CommentTok{2021{-}12{-}28}
\CommentTok{*/}
\KeywordTok{import} \ImportTok{javax}\OperatorTok{.}\ImportTok{swing}\OperatorTok{.}\ImportTok{JOptionPane}\OperatorTok{;}
\KeywordTok{import} \ImportTok{java}\OperatorTok{.}\ImportTok{util}\OperatorTok{.}\ImportTok{regex}\OperatorTok{.*;}

\KeywordTok{public} \KeywordTok{class}\NormalTok{ CalcTaxes }\OperatorTok{\{}
    \KeywordTok{public} \DataTypeTok{static} \DataTypeTok{void} \FunctionTok{main}\OperatorTok{(}\BuiltInTok{String}\OperatorTok{[]}\NormalTok{ args}\OperatorTok{)} \OperatorTok{\{}
        \DataTypeTok{boolean}\NormalTok{ calcAgain }\OperatorTok{=} \KeywordTok{true}\OperatorTok{;} \CommentTok{//control to run again or exit}

        \BuiltInTok{JOptionPane}\OperatorTok{.}\FunctionTok{showMessageDialog}\OperatorTok{(}
            \KeywordTok{null}\OperatorTok{,} \StringTok{"This program will calculate your federal and state taxes"}\OperatorTok{);}

        \ControlFlowTok{while} \OperatorTok{(}\NormalTok{calcAgain}\OperatorTok{)} \OperatorTok{\{}
            \BuiltInTok{String}\NormalTok{ validationPattern }\OperatorTok{=} \KeywordTok{null}\OperatorTok{;}
            \BuiltInTok{String}\NormalTok{ errorMessage }\OperatorTok{=} \KeywordTok{null}\OperatorTok{;}
            \BuiltInTok{String} \OperatorTok{[]}\NormalTok{ userInputs }\OperatorTok{=} \KeywordTok{new} \BuiltInTok{String}\OperatorTok{[}\DecValTok{4}\OperatorTok{];} \CommentTok{//array to hold user input strings}
            \BuiltInTok{String} \OperatorTok{[][]}\NormalTok{ inputRequests }\OperatorTok{=} \OperatorTok{\{} \CommentTok{//input requests and expected type (num or str)}
                \OperatorTok{\{}\StringTok{"Please enter your name:"}\OperatorTok{,}\StringTok{"str"}\OperatorTok{\},}
                \OperatorTok{\{}\StringTok{"Enter your yearly income"}\OperatorTok{,}\StringTok{"num"}\OperatorTok{\},}
                \OperatorTok{\{}\StringTok{"Enter your Federal tax rate (\%):"}\OperatorTok{,}\StringTok{"num"}\OperatorTok{\},}
                \OperatorTok{\{}\StringTok{"Enter your State tax rate (\%):"}\OperatorTok{,}\StringTok{"num"}\OperatorTok{\}}
            \OperatorTok{\};}
            \ControlFlowTok{for} \OperatorTok{(}\DataTypeTok{int}\NormalTok{ i}\OperatorTok{=}\DecValTok{0}\OperatorTok{;}\NormalTok{ i }\OperatorTok{\textless{}}\NormalTok{ inputRequests}\OperatorTok{.}\FunctionTok{length}\OperatorTok{;}\NormalTok{ i}\OperatorTok{++)} \OperatorTok{\{}
\NormalTok{                userInputs}\OperatorTok{[}\NormalTok{i}\OperatorTok{]} \OperatorTok{=} \BuiltInTok{JOptionPane}\OperatorTok{.}\FunctionTok{showInputDialog}\OperatorTok{(}\NormalTok{inputRequests}\OperatorTok{[}\NormalTok{i}\OperatorTok{][}\DecValTok{0}\OperatorTok{]);}
                \ControlFlowTok{if} \OperatorTok{(}\NormalTok{userInputs}\OperatorTok{[}\NormalTok{i}\OperatorTok{]} \OperatorTok{==} \KeywordTok{null}\OperatorTok{)} \OperatorTok{\{} \CommentTok{// Exit cleanly if user hits cancel}
                    \BuiltInTok{System}\OperatorTok{.}\FunctionTok{exit}\OperatorTok{(}\DecValTok{0}\OperatorTok{);}
                \OperatorTok{\}}
                \ControlFlowTok{else} \ControlFlowTok{if} \OperatorTok{(}\NormalTok{inputRequests}\OperatorTok{[}\NormalTok{i}\OperatorTok{][}\DecValTok{1}\OperatorTok{].}\FunctionTok{equals}\OperatorTok{(}\StringTok{"num"}\OperatorTok{))\{}
\NormalTok{                    validationPattern }\OperatorTok{=} \StringTok{"\^{}}\SpecialCharTok{\textbackslash{}\textbackslash{}}\StringTok{d+$|\^{}{-}?}\SpecialCharTok{\textbackslash{}\textbackslash{}}\StringTok{d*}\SpecialCharTok{\textbackslash{}\textbackslash{}}\StringTok{.}\SpecialCharTok{\textbackslash{}\textbackslash{}}\StringTok{d\{2\}$"}\OperatorTok{;} \CommentTok{// match integer or 2 place decimal}
\NormalTok{                    errorMessage }\OperatorTok{=} \StringTok{"Please enter a number }\SpecialCharTok{\textbackslash{}n}\StringTok{(000 or 00.00)"}\OperatorTok{;}
                \OperatorTok{\}}
                \ControlFlowTok{else} \ControlFlowTok{if} \OperatorTok{(}\NormalTok{inputRequests}\OperatorTok{[}\NormalTok{i}\OperatorTok{][}\DecValTok{1}\OperatorTok{].}\FunctionTok{equals}\OperatorTok{(}\StringTok{"str"}\OperatorTok{))} \OperatorTok{\{}
\NormalTok{                    validationPattern }\OperatorTok{=} \StringTok{"\^{}[A{-}Za{-}z]+}\SpecialCharTok{\textbackslash{}\textbackslash{}}\StringTok{s*[A{-}Za{-}z]*$"}\OperatorTok{;} \CommentTok{// match name and optional 2nd name}
\NormalTok{                    errorMessage }\OperatorTok{=} \StringTok{"Try again./nNo numbers or symbols please"}\OperatorTok{;}
                \OperatorTok{\}}
                \ControlFlowTok{else} \OperatorTok{\{}
                    \BuiltInTok{JOptionPane}\OperatorTok{.}\FunctionTok{showMessageDialog}\OperatorTok{(}
                        \KeywordTok{null}\OperatorTok{,} \StringTok{"Illegal Input type Option. Please contact the developer."}\OperatorTok{);}
                    \ControlFlowTok{throw} \KeywordTok{new} \BuiltInTok{IllegalArgumentException}\OperatorTok{(}
                        \StringTok{"Input type option invalid. Only \textquotesingle{}num\textquotesingle{} or \textquotesingle{}str\textquotesingle{} allowed"}\OperatorTok{);}
                \OperatorTok{\}}

                \BuiltInTok{Pattern}\NormalTok{ p }\OperatorTok{=} \BuiltInTok{Pattern}\OperatorTok{.}\FunctionTok{compile}\OperatorTok{(}\NormalTok{validationPattern}\OperatorTok{);}
                \BuiltInTok{Matcher}\NormalTok{ m }\OperatorTok{=}\NormalTok{ p}\OperatorTok{.}\FunctionTok{matcher}\OperatorTok{(}\NormalTok{userInputs}\OperatorTok{[}\NormalTok{i}\OperatorTok{]);}
                \ControlFlowTok{if} \OperatorTok{(!}\NormalTok{m}\OperatorTok{.}\FunctionTok{find}\OperatorTok{())\{} 
                    \BuiltInTok{JOptionPane}\OperatorTok{.}\FunctionTok{showMessageDialog}\OperatorTok{(}\KeywordTok{null}\OperatorTok{,}\NormalTok{ errorMessage}\OperatorTok{);}
\NormalTok{                    i}\OperatorTok{{-}{-};}
                \OperatorTok{\}}
            \OperatorTok{\}}

            \DataTypeTok{double}\NormalTok{ yearlyIncome }\OperatorTok{=} \BuiltInTok{Double}\OperatorTok{.}\FunctionTok{parseDouble}\OperatorTok{(}\NormalTok{userInputs}\OperatorTok{[}\DecValTok{1}\OperatorTok{]);}
            \DataTypeTok{double}\NormalTok{ fedTaxRate }\OperatorTok{=} \BuiltInTok{Double}\OperatorTok{.}\FunctionTok{parseDouble}\OperatorTok{(}\NormalTok{userInputs}\OperatorTok{[}\DecValTok{2}\OperatorTok{])} \OperatorTok{/} \DecValTok{100}\OperatorTok{;}
            \DataTypeTok{double}\NormalTok{ stateTaxRate }\OperatorTok{=} \BuiltInTok{Double}\OperatorTok{.}\FunctionTok{parseDouble}\OperatorTok{(}\NormalTok{userInputs}\OperatorTok{[}\DecValTok{3}\OperatorTok{])} \OperatorTok{/} \DecValTok{100}\OperatorTok{;}
            \DataTypeTok{double}\NormalTok{ fedTaxDue }\OperatorTok{=}\NormalTok{ yearlyIncome }\OperatorTok{*}\NormalTok{ fedTaxRate}\OperatorTok{;}
            \DataTypeTok{double}\NormalTok{ stateTaxDue }\OperatorTok{=}\NormalTok{ yearlyIncome }\OperatorTok{*}\NormalTok{ stateTaxRate}\OperatorTok{;}

            \BuiltInTok{JOptionPane}\OperatorTok{.}\FunctionTok{showMessageDialog}\OperatorTok{(}\KeywordTok{null}\OperatorTok{,}\NormalTok{ userInputs}\OperatorTok{[}\DecValTok{0}\OperatorTok{]} \OperatorTok{+} \StringTok{"}\SpecialCharTok{\textbackslash{}n}\StringTok{Your Federal Taxes are: $"} 
                \OperatorTok{+}\NormalTok{ fedTaxDue }\OperatorTok{+} \StringTok{"}\SpecialCharTok{\textbackslash{}n}\StringTok{Your State taxes are: $"} \OperatorTok{+}\NormalTok{ stateTaxDue}\OperatorTok{);}

            \DataTypeTok{int}\NormalTok{ reply }\OperatorTok{=} \BuiltInTok{JOptionPane}\OperatorTok{.}\FunctionTok{showConfirmDialog}\OperatorTok{(}\KeywordTok{null}\OperatorTok{,} \StringTok{"Would you like to calculate more?"}\OperatorTok{,} 
                \StringTok{"Run Again?"}\OperatorTok{,}  \BuiltInTok{JOptionPane}\OperatorTok{.}\FunctionTok{YES\_NO\_OPTION}\OperatorTok{);}
            \ControlFlowTok{if} \OperatorTok{(}\NormalTok{reply }\OperatorTok{==} \BuiltInTok{JOptionPane}\OperatorTok{.}\FunctionTok{NO\_OPTION}\OperatorTok{)} \OperatorTok{\{}
\NormalTok{                    calcAgain }\OperatorTok{=} \KeywordTok{false}\OperatorTok{;}
            \OperatorTok{\}}
        \OperatorTok{\}}
    \OperatorTok{\}} 
\OperatorTok{\}}
\end{Highlighting}
\end{Shaded}

\hypertarget{user-screens}{%
\section{User Screens}\label{user-screens}}

\begin{figure}
\hypertarget{fig:screen1}{%
\centering
\includegraphics[width=\textwidth,height=1.5in]{media/CSC310-Case1-1.png}
\caption{CalcTaxes Screen 1}\label{fig:screen1}
}
\end{figure}

\begin{figure}
\hypertarget{fig:screen2}{%
\centering
\includegraphics[width=\textwidth,height=1.5in]{media/CSC310-Case1-2.png}
\caption{CalcTaxes Screen 2}\label{fig:screen2}
}
\end{figure}

\begin{figure}
\hypertarget{fig:screen3}{%
\centering
\includegraphics[width=\textwidth,height=1.5in]{media/CSC310-Case1-3.png}
\caption{CalcTaxes Screen 3}\label{fig:screen3}
}
\end{figure}

\begin{figure}
\hypertarget{fig:screen4}{%
\centering
\includegraphics[width=\textwidth,height=1.5in]{media/CSC310-Case1-4.png}
\caption{CalcTaxes Screen 4}\label{fig:screen4}
}
\end{figure}

\begin{figure}
\hypertarget{fig:screen5}{%
\centering
\includegraphics[width=\textwidth,height=1.5in]{media/CSC310-Case1-5.png}
\caption{CalcTaxes Screen 5}\label{fig:screen5}
}
\end{figure}

\begin{figure}
\hypertarget{fig:screen6}{%
\centering
\includegraphics[width=\textwidth,height=1.5in]{media/CSC310-Case1-6.png}
\caption{CalcTaxes Screen 6}\label{fig:screen6}
}
\end{figure}





\end{document}